\documentclass{article}

\usepackage{graphicx}

\title{Data Mining, Homework \#1}
\author{Giacomo Lanciano (1487019) \and Fabio Rosato (1565173)}
\date{November 13, 2016}

\begin{document}

\maketitle
\tableofcontents
\newpage

\section{Problem 1}
We shuffle a standard deck of cards, obtaining a permutation that is uniform over all
52! possible permutations.
\begin{itemize}
	\item[1.] Define a proper probability space $\Omega$ for the above random process. What is the probability of each element in $\Omega$?
	\item[2.] Find the probability of the following events:
	\begin{itemize}
		\item[(a)] The first three cards include at least one ace.
		\item[(b)] The first five cards include exactly one ace.
		\item[(c)] The first three cards are a pair of the same rank (they are the same number or both are
		J, or both are Q, etc.)
		\item[(d)] The first five cards are all diamonds.
		\item[(e)] The first five cards form a full house (three of one rank and two of another rank).		
	\end{itemize}
	\item[3.] (Optional) Develop some small programs in Python to perform simulations to check your answers.
\end{itemize}

\subsection{Solution}

\newpage
\section{Problem 2}
You throw a set of 3 regular dice again and again, until for the first time you see a
sum of 11 or a sum of 16.
\begin{itemize}
	\item[1.] Design an appropriate probability space for the above process.
	\item[2.] What is the probability that you stop because you see a sum of 16?
\end{itemize}

\subsection{Solution}

\newpage
\section{Problem 3}
A group of n man and m women go to a Chinese restaurant and sit in a round table,
such that each person has to other person next to him/her.
\begin{itemize}
	\item[1.] Describe a sample space that describes the random process.
	\item[2.] Find the expected number of men who will be sitted next to at least one woman.
\end{itemize} 

\subsection{Solution}

\newpage
\section{Problem 4}
Blah blah\ldots


\end{document}
