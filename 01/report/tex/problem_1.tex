% !TeX spellcheck = en_GB

\section{Problem 1}

We shuffle a standard deck of cards, obtaining a permutation that is uniform over all
52! possible permutations.
\begin{itemize}
	\item[1.] Define a proper probability space $\Omega$ for the above random process. What is the probability of each element in $\Omega$?
	\item[2.] Find the probability of the following events:
	\begin{itemize}
		\item[(a)] The first three cards include at least one ace.
		\item[(b)] The first five cards include exactly one ace.
		\item[(c)] The first three cards are a pair of the same rank (they are the same number or both are J, or both are Q, etc.)
		\item[(d)] The first five cards are all diamonds.
		\item[(e)] The first five cards form a full house (three of one rank and two of another rank).		
	\end{itemize}
	\item[3.] (Optional) Develop some small programs in Python to perform simulations to check your answers.
\end{itemize}


\subsection{Probability space definition}

Let $S = \{h, d, c, s\}$ be the set of card suits. Let $R = \{A, 2, \ldots, 10, J, Q, K\}$ be the set of card ranks. Hence, $C = S \times R$ is the set of cards contained in a French deck. Let $\Omega$ be the probability space of this process. Therefore,
\begin{align*}
	\Omega &= Sym(C)\\
		&= \text{symmetric group of C}\\
		&= \text{collection of all permutations of the elements in C}
\end{align*}
By definition of symmetric group\cite{sym_group}, we have that
\begin{align*}
	|\Omega| = |C|! = (|S| \cdot |R|)! = 52!
\end{align*}
In addition, we have that $\Omega$ is an equiprobable space, since every permutation of $C$ has the same probability to be the result of a shuffle. Hence,
\begin{align*}
	\Pr(\omega) = \dfrac{1}{|\Omega|} = \dfrac{1}{52!} \qquad \forall \ \omega\in\Omega
\end{align*}


\subsection{Events probabilities}
\begin{itemize}
	\item[(a)] Let $A_{ij} \subset \Omega$ be the event related to drawing exactly $i$ aces in the first $j$ cards of the deck $(0 \leq i \leq 4 \ , \ j \geq i)$. Therefore,
	\begin{align}
		\Pr(A_{ij}) = \dfrac{\binom{4}{i} \cdot \binom{48}{j-i} \cdot j! \cdot (52-j)!}{52!} \label{pr_aij}
	\end{align}
	
	\newpage
	since there are:
	\begin{itemize}
		\item $\binom{4}{i}$ ways to choose $i$ aces.
		\item $\binom{48}{j-i}$ ways to chose the remaining $j-i$ cards.
		\item $j!$ permutations for each of the first $j$ cards combination.
		\item $52-j!$ permutations of the remaining cards.
	\end{itemize}
	
	Let $A \subset \Omega$ be the event related to drawing at least 1 ace in the first 3 cards. Therefore,
	\begin{align}
		\Pr(A) = \Pr(A_{13} \cup A_{23} \cup A_{33}) = \Pr(A_{13}) + \Pr(A_{23}) + \Pr(A_{33}) \label{pr_a}
	\end{align}
	
	Note that we can write (\ref{pr_a}) since the events $A_{13}, A_{23}$ and $A_{33}$ are pairwise disjoint. Using (\ref{pr_aij}), we can easily compute the single probabilities and conclude that
	\begin{align*}
		\Pr(A) \simeq 0.217
	\end{align*}
	
	\item[(b)] Let $B \subset \Omega$ be the event related to drawing exactly 1 ace in the first 5 cards. Therefore,
	\begin{align*}
		\Pr(B) = \Pr(A_{15}) \simeq 0.299
	\end{align*}
	
	reusing (\ref{pr_aij}).
	
	\item[(c)] Let $C \subset \Omega$ be the event related to drawing a "tris" in the first 3 cards. Therefore,
	\begin{align*}
		\Pr(C) = \dfrac{13 \cdot \binom{4}{3} \cdot 3! \cdot 49!}{52!} \simeq 0.0023
	\end{align*}
	
	since there are:
	\begin{itemize}
		\item 13 different ranks.
		\item $\binom{4}{3}$ ways to choose 3 cards among those with the same rank.
		\item $3!$ permutations for each of the first $3$ cards combination.
		\item $49!$ permutations of the remaining cards.
	\end{itemize}
	
	\item[(d)] Let $D \subset \Omega$ be the event related to drawing only Diamonds in the first 5 cards. Therefore,
	\begin{align*}
		\Pr(D) = \dfrac{\binom{13}{5} \cdot 5! \cdot 47!}{52!} \simeq 0.0005
	\end{align*}
	
	since there are:
	\begin{itemize}
		\item $\binom{13}{5}$ ways to choose $5$ cards among the Diamonds.
		\item $5!$ permutations for each of the first $5$ cards combination.
		\item $47!$ permutations of the remaining cards.
	\end{itemize}
	
	\item[(e)] Let $E \subset \Omega$ be the event related to drawing a "full house" in the first 5 cards. Therefore,
	\begin{align*}
		\Pr(E) = \dfrac{13 \cdot 12 \cdot \binom{4}{2} \cdot \binom{4}{3} \cdot 5! \cdot 47!}{52!} \simeq 0.0014
	\end{align*}
	since there are:
	\begin{itemize}
		\item $13 \cdot 12$ different dispositions of 2 (out of 13) ranks.
		\item $\binom{4}{2}$ ways to choose the suits of the pair.
		\item $\binom{4}{3}$ ways to choose the suits of the tris.
		\item $5!$ permutations for each of the first $5$ cards combination.
		\item $47!$ permutations of the remaining cards.
	\end{itemize}		
\end{itemize}


\subsection{Event simulation in Python}
We developed a simple Python script that performs drawing simulations to check our previous answers. You can find the source code in \textit{deck.py} (see attachments).
We used the Python module \textit{pydealer}\cite{pydealer} to simulate the drawing of 5 cards from a deck of French cards. Basically, we repeated the drawing process many times (e.g. a million trials), keeping track of how many times each event occurred.
In the end we computed the frequency of each event, dividing its occurrences by the total number of trials.
Results for a million trials are shown below.
\begin{center}
	\vspace{5mm}
	\includegraphics[scale=0.8]{img/deck.jpg}
\end{center}
