% !TeX spellcheck = en_GB

\section{Problem 4}

We are monitoring a graph, which arrives as a stream of edges $\mathcal{E} = e_1, e_2, \ldots$. We
assume that exactly one edge arrives at a time, with edge $e_{i}$ arriving at time $i$, and the stream is starting at time 1. Each edge $e_{i}$ is a pair of vertices $(u_{i},v_{i})$, and we use $V$ to denote the set of all vertices that we have seen so far.\\
We assume that we are working in the sliding window model. According to that model, at each
time $t$ only the $w$ most recent edges are considered active. Thus, the set of active edges $E(t,w)$ at time $t$ and for window length $w$ is:
\begin{align*}
E(t,w) = 
	\left\{ \begin{aligned}
		&e_{t-w+1},\ldots,e_t &\quad\text{if }t > w\\
		&e_1,\ldots,e_t &\quad\text{if }t \le  w
	\end{aligned}\right.
\end{align*}
We then write $G(t,w) = (V,E(t,w))$ to denote the graph that consists of the active edges at time $t$, given a window length $w$.\\
As an example, given the stream of edges:
\begin{align*}
	e1 = (c,e), e2 = (b,d), e3 = (a,c), e4 = (c,b), e5 = (a,b), e6 = (c,d), e7 = (d,e)
\end{align*}
the graphs $G(5,5)$, $G(6,5)$ and $G(7,5)$ are shown below:
\begin{center}
	\includegraphics[scale=0.5]{img/example_graphs.jpg}
\end{center}
Notice that for all $t \ge w$ the graph $G(t+1,w)$ results from $G(t,w)$ by adding one edge and deleting one edge.\\
We want to monitor the connectivity of the graph $G(t,w)$. In other words, we want to design an
algorithm that quickly decides, at any time $t$, if the graph $G(t,w)$ is connected. In the previous example, the graphs $G(5,5)$ and $G(7,5)$ are connected, while the graph $G(6,5)$ is not connected.
\begin{enumerate}
	\item Propose a streaming algorithm for deciding the connectivity of $G(t,w)$.\\
	\textbf{Hint:} An efficient streaming algorithm takes advantage of the fact that the graph $G(t+1,w)$ changes very little compared to $G(t,w)$. Therefore, our algorithm should be able to efficiently update the connectivity of $G(t + 1,w)$ when a new edge $e_{t+1}$ arrives at time $t + 1$, given that the connectivity of $G(t,w)$ has already been computed.
	\item How much space does your algorithm use? Try to design an algorithm that manages to use
	$\mathcal{O}(n)$ space, where $n = |V|$.
	\item What is the update time of your algorithm?\\
	\textbf{Hint for 2 and 3:} The space of your algorithm is the maximum amount of space used at
	any given moment. The update time is the time needed to compute the output at time $t+1$,
	given the state at time $t$ and the new edge $e_{t+1}$ that arrives at time $t + 1$. You should provide your answer using the $\mathcal{O}(\cdot)$ notation, written as a function of the window length $w$ and the number of vertices $n$.
\end{enumerate}

\subsection{Streaming connectivity checking}

A \textbf{naive} approach for this problem could be performing a visit of the graph (DFS or BFS) each time the window slides forward, to check whether the number of visited vertices is equal to $|V|$ or not. However, this simple algorithm has a \textbf{space} complexity equal to $\mathcal{O}(w)$, since all the edges in window have to be kept in memory to perform the check, that is infeasible in these settings.\\
Our idea is based on the fact that graph connectivity can be checked looking only at a \textbf{spanning tree} of $G$. Let $ST(G)$ be a spanning tree of $G$, then
\begin{align*}
ST(G)\text{ is connected }\Leftrightarrow \ G\text{ is connected}
\end{align*}
In addition, in this case, we want to consider $ST(G)$ such that it is made of the \textbf{most recent} edges in the stream, to delay its updates as much as possible. Note that \textbf{checking connectivity} on $ST(G)$ has \textbf{space} complexity equal to 
\begin{align*}
\mathcal{O}(|V|)
\end{align*}
since we simply need to perform a visit of the tree, whose number of edges is $|V| - 1$.\\
Clearly, we also need to take into account the \textbf{time} complexity of managing the spanning tree. Again, a naive approach could be to rebuild it each time the window slides forward, but it would cost $\mathcal{O}(w)$ (infeasible). However, we could simply \textbf{update} it whenever the window moves, with a cost equal to $\mathcal{O}(|V|)$. A simple algorithm to do that is:
\begin{enumerate}
	\item Add newly read edge to $ST(G)$.
	\item Detect the \textbf{loop} possibly introduced in step 1 (we can use \textbf{Algorithm \ref{loop_check}} to do that) and delete the oldest edge in it.
\end{enumerate}
Note that, since the graph trivially have \textbf{only one} loop, we have that $|V| = |E|$. Hence, \textbf{time} complexity of \textbf{Algorithm \ref{loop_check}} (in this case), and thus of our \textbf{update algorithm}, is
\begin{align*}
\mathcal{O}(|V|)
\end{align*}
What follows is the pseudocode of our streaming algorithm to decide whether $G(t,w)$ is connected.

\newpage
\begin{algorithm}
	\caption{Check $G(t,w)$ connectivity}
	\begin{algorithmic}[1]
		\State $st \ \leftarrow \ \text{empty spanning tree}$
		\Event{new edge $e_t = (u,v)$ in stream}
			\State $st.add(e_t)$
			\State $st.removeExpiredEdge()$
			\State $st.removeOldestEdgeInCycle()$
			\State \Return $st.checkConnectivity()$
		\EndEvent
	\end{algorithmic}
\end{algorithm}
